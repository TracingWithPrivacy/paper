\documentclass{article}
% \documentclass[twocolumn]{article}
% \usepackage{tikz-cd}
\usepackage{amsmath,amsfonts,amsthm,amssymb}
\usepackage{mathtools}
\usepackage{yfonts}
\DeclarePairedDelimiter\ceil{\lceil }{\rceil}
\DeclarePairedDelimiter\floor{\lfloor }{\rfloor}

\title{GeiGovid}
% \author{michaendr}

\begin{document}
\maketitle
\begin{abstract}
  Airborne infectious diseases are an invisible threat.
  This paper sketches an infectious diseases geiger counter that 'runs ahead' of
  the disease, revealing potentially dangerous exposure
  to infectious diseases while respecting privacy.
\end{abstract}

\section{Overview}


The idea is to transmit a probabilistic measure of exposure by Bluetooth beacons, upon
which sensible estimates can be given for how likely a person is to be infected.


Exposing such a tangible, roughly accurate measure for the exposure
may increase awareness and adherence to social distancing.


The next section illustrates a rough derivation of what such a
probability estimate could look like:

Simplifying assumptions are made: The infection is 'memoryless' in that after a
person is
infected, exposure to other infected hosts does not change the prognosis of
the disease.

\section{Infection model}

In the following we first build a model in discrete time of infection
probability from the perspective of a single person \(P\), assuming our infection
can be modelled as independent Bernoulli trials that are not necessarily
identically distributed. In order to do this we pick a timeframe of interest that
spans from some time in the past \(t-T\) to the current time \(t\),
outside of which we assume no infection could occur. This timeframe is
subdivided into \(n\) intervals:
\begin{align*}
  k \in \{1,...,n+1\}: t_k &= t-T + \frac{k-1}{n-1}T\\
                           \delta t &= t_{k+1} -t_k
\end{align*}
Our probability space is thus
\begin{align*}
  \mathcal{X} &= \{0,1\}^n\\
  \mathcal{A} &= 2^\mathcal{X}\\
  p_k \in [0,1]:\ \mathbb{P}((x_1,...,x_n)) &= \prod_{k=1}^n (x_k p_k + (1-x_k)(1-p_k))
\end{align*}
where \(x_k=1\) means an infection occurred in the interval \(t_k\) and \(p_k\)
is the probability of infection in that interval.
\par
This probability itself can be expressed as the probability of any nearby people
infecting our person of interest, which is another sequence of Bernoulli trials
which are independent, not necessarily identically distributed. \(l\) indexes
all other people, in practice of course most of which will not be able to infect
\(P\) due to distance:
\begin{align*}
  p_k &= \mathbb{P}'\left( \bigcup_{l=1}^N \overbrace{A_{kl}}^{l\ \text{infects}\ P\ \text{at interval}\ k}\right)\\
      &= 1 - \mathbb{P}'\left( \bigcap_{l=1}^N A_{kl}^c\right)\\
      &= 1 - \prod_{l=1}^N {\mathbb{P}'\left( A_{kl}^c \right)}\\
      &= 1 - \prod_{l=1}^N {(1-p_{kl})}
\end{align*}
Assuming nice enough functions with sufficient differentiability
\[
  p_{kl} = p_l[k] = \mathcal{F}_l\left(t_{k+1}\right)- \mathcal{F}_l\left(t_k\right)
\]
Note \(\mathcal{F}\) isn't a cumulative distribution function, as it
doesn't always hold that \(\mathcal{F}(t) = 1\), however monotonicity and
positivity do hold.
\[
  \lim_{\delta t \rightarrow 0} \frac{\mathcal{F}_l\left(t_k + \delta t\right)- \mathcal{F}_l\left(t_k\right)}{\delta t} = \mathcal{P}_l\left(t_k\right)
\]
Thus for sufficiently small \(\delta t\) it holds that
\begin{align*}
  p_{kl} \approx \mathcal{P}_l(t_k)\delta t
\end{align*}
% For speculation on possible decompositions of \(\mathcal{P}(t)\) see appendix.

\section{Probability of infection}

\begin{align}
  \mathbb{P}(\text{not infected}) &= \prod_{j=1}^n (1-p_j)\\
                                  &= \prod_{j=1}^n \prod_{k=1}^N (1-p_{jk})\\
                                  &= \prod_{k=1}^N \prod_{j=1}^n (1-p_{jk})\\
                                  &= \prod_{k=1}^N \prod_{j=1}^n \left(1-\mathcal{P}_k\left(t_j\right)\delta t + \mathcal{O}\left(\delta t^2\right)\right)\\
                                  &= \prod_{k=1}^N \exp\left( \sum_{j=1}^n \ln \left(1-\mathcal{P}_k\left(t_j\right)\delta t + \mathcal{O}\left(\delta t^2\right)\right) \right)
\end{align}

\noindent
Since
\[
  \lvert x \rvert \leq D < 1: \ln (1-x) = - \sum_{l=1}^\infty \frac{x^l}{l}
\]
for sufficiently small \(\delta t\) we have absolute convergence
\begin{align*}
  \sum_{j=1}^n \ln \left(1-\mathcal{P}_k\left(t_j\right)\delta t + \mathcal{O}\left(\delta t^2\right)\right) &= -\sum_{j=1}^n \sum_{l=1}^\infty \frac{\left( \mathcal{P}_k\left(t_j\right)\delta t + \mathcal{O}\left(\delta t^2\right)\right)^l}{l}\\
                                                                                                             &=-\sum_{j=1}^n \left( \mathcal{P}_k\left( t_j \right) \delta t + \mathcal{O}\left( \delta t^2 \right)\right)\\
  &\overset{\delta t \rightarrow 0}{=} - \int_{t-T}^t \mathcal{P}_k(\tau)\, d\tau
\end{align*}
Inserting back into equation \(5\) gives
\[
  \mathbb{P}(\text{not infected}) = \exp \left( - \int_{t-T}^t \sum_{k=1}^N
    \mathcal{P}_k(\tau)\,d\tau\right)
\]
\section{Connecting infectiousness with probability of infection}

We'll use the expectation of an infectiousness random variable as our estimate
of current infectiousness, again first in discrete time:
\begin{align*}
  \mathfrak{i} &= E\left[ I[n] \right]\\
                 &= E\left[ E\left[ I[n] \vert S = j \right] \right]
\end{align*}
Here \(S\) denotes the time of infection.\\
Expand into
\begin{align*}
 \sum_{j=1}^n E\left[ I[n] \vert S = j \right] \cdot \mathbb{P}(S = j)
\end{align*}
Now we use the assumption that the prognosis of disease is independent of
subsequent infections, so only the first counts, from then on which we know from
statistical data how the expectation of \(I\) develops:
\begin{align}
  \sum_{j=1}^n E\left[ I[n] \vert S = j \right] \cdot \mathbb{P}(S = j) &= \sum_{j=1}^n \mathcal{I}\left( t_n-t_j \right) \cdot \mathbb{P}(S = j)
\end{align}
The probability that the infection started at \(j\) means all prior Bernoulli trials evaluated
to \(0\):
\begin{align*}
  \mathbb{P}(S = j) &= p_j \prod_{k=1}^{j-1}(1-p_k)\\
                    &= \left( \overbrace{\sum_{l=1}^N \mathcal{P}_l\left(t_j\right)}^{\theta \left(t_j\right)}\delta t
                      + \mathcal{O}\left(\delta t^2\right)\right)\prod_{k=1}^{j-1} \left( 1-\sum_{l=1}^N\mathcal{P}_l\left( t_k \right)\delta t + \mathcal{O}\left( \delta t^2 \right)\right)\\
                    &= \left( \theta\left( t_j \right) \delta t + \mathcal{O}\left( \delta t^2 \right) \right)
                      \exp \left( \sum_{k=1}^{j-1} \ln \left( 1- \theta\left( t_k \right)\delta t
                      + \mathcal{O}\left(\delta t^2 \right)\right)\right)\\
                    &= \left( \theta\left( t_j \right) \delta t + \mathcal{O}\left( \delta t^2 \right) \right)
                      \exp \left( -\sum_{k=1}^{j-1} \left( \theta\left( t_k \right)\delta t
                      + \mathcal{O}\left(\delta t^2 \right) \right) \right)\\
                    &= \left( \theta\left( t_j \right) \delta t + \mathcal{O}\left( \delta t^2 \right) \right)
                      \exp \left( -\sum_{k=1}^{j-1} \theta\left( t_k \right)\delta t
                      + \mathcal{O}\left(\delta t \right) \right)\\
                    &= \left( \theta\left( t_j \right) \delta t + \mathcal{O}\left( \delta t^2 \right) \right)
                      \left( \exp \left( -\sum_{k=1}^{j-1} \theta\left( t_k \right)\delta t
                      \right)+ \mathcal{O}\left(\delta t \right)  \right)\\
                    &= \theta\left( t_j \right)
                      \exp \left( -\sum_{k=1}^{j-1} \theta\left( t_k \right)\delta t
                      \right)\delta t+  \mathcal{O}\left( \delta t^2\right)
\end{align*}
Inserting into equation \(6\) gives
\begin{align*}
  \sum_{j=1}^n \mathcal{I}\left( t_n-t_j \right) \theta\left( t_j \right)
  \exp \left( -\sum_{k=1}^{j-1} \theta\left( t_k \right)\delta t
  \right)\delta t+  \mathcal{O}\left( \delta t\right)
\end{align*}
\[
  \int_{t-T}^t \mathcal{I}(t-\tau)\theta(\tau)\exp \left( -\int_{t-T}^\tau \theta\left( \tau' \right)d\tau'\right)d\tau
\]
\section{Privacy issues}

None beyond any introduced by Bluetooth.

\section{Robustness against DoS}

DoS can be attempted in three ways:
An attacker can choose to broadcast a different estimator, either higher or lower than the protocol
would suggest.

An estimate that is broadcasted cannot be exceedingly high as this risks being
detected as out of the ordinary.
The more serious attack is thus to send inconspicuous estimates, yet
fake a large number of devices. This seems not entirely riskless, but plausible.

The other option is to send an estimate that is too low. No privacy-preserving
system that protects participants from being identified against their will can
prevent this. The worst thing an attacker can do in this case is
to not participate at all. This is acceptable. Unacceptable would be an attacker
able to disrupt the operation of the system for honest users.

\section{Benefits}
\begin{itemize}
\item Easy to implement
\item Low footprint
\item Immunity can be advertised with a zero estimator
\end{itemize}

\section{Downsides}
\begin{itemize}
\item Technology-tyranny, opaque to end user
  \begin{itemize}
  \item Correctness of contact tracing easy to demonstrate
  \end{itemize}
\item Relies on statistical information that may be unavailable
\item Needs external jump start
\item Might need manual intervention if model gets out of hand
\end{itemize}

\section{Extensions}
An extension could try closing the feedback loop to the health authority in a
privacy-respecting way, to convey information which can be used to refine the
model. One half of the loop is to retrieve updated models from the health
authority and updating prior information. Adding model version
to the beacon could add accuracy, but complicates reasoning about
attack vectors. Less accurate, but more robust would be to assume information
collected is using the most up to date model
available from the health authority.

\section{Caveats}

Not reviewed by anybody with experience in infectious diseases. Or any other expert.

\section*{Appendix}


% And more boldly perhaps that \(p_{kl}\) can be factored into functions one of
% which is only dependent on the distance, and the other being a universal
% normalised measure of virulence:
% \begin{align*}
%   p_{kl} = p_l[k] = D\left(d_l[k]\right)\mathcal{I}_l\left(t_k\right)
% \end{align*}

\end{document}
